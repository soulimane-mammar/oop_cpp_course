\documentclass[10pt,a4paper]{article}
\usepackage[utf8]{inputenc}
\usepackage[T1]{fontenc}
\usepackage{amsmath}
\usepackage{amssymb}
\usepackage{graphicx}
\title{Assignment \#2}
\author{Soulimane Mammar}
\begin{document}
	\maketitle
	\section*{Exercise 1}
	Write a program that helps a person decide whether to buy a hybrid car. Your
	program’s inputs should be:
	\begin{itemize}
		\item The cost of a new car
		\item The estimated kilometers driven per year
		\item The estimated gas price
		\item The estimated resale value after 5 year
	\end{itemize}
	Compute the total cost of owning the car for 5 years. Obtain realistic prices for a new and used hybrid and a comparable car from the Web. Run your
	program twice, using today’s gas price and 15,000 kilometers per year. Include code and the program runs with your assignment.
	
	\section*{Exercise 2}
	The following pseudocode describes how a bookstore computes the price of an order from the total price and the number of the books that were ordered.
	\begin{enumerate}
		\item Read the total book price and the number of books.
		\item Compute the tax (7.5\% of the total book price).
		\item Compute the shipping charge (50 DZD per book).
		\item The price of the order is the sum of the total book price, the tax, and the shipping charge. Print the price of the order.
	\end{enumerate}
	Translate this pseudocode into a C++ program.
	
	\section*{Exercise 3}
	Giving change. Implement a program that directs a cashier how to give change. The program has two inputs: the	amount due and the amount received from the customer.
	Display the five, ten , twenty , fifty, hundred, two hundreds, five hundreds and a thousand dinars that the customer should receive in return.
	
	\section*{Exercise 4}
	File names and extensions. Write a program that prompts the user for the drive letter (\verb|C|), the path (\verb|\Windows\System|), the file name (\verb|Readme|), and the extension (\verb|txt|). Then	print the complete file name \verb|C:\Windows\System\Readme.txt|.
	
	\section*{Exercise 5}
	Write a program that reads an integer and breaks it into a sequence of individual
	digits. For example, the input \verb|16384| is displayed as
	
	\verb|1 6 3 8 4|

	You may assume that the input is not negative.
\end{document}