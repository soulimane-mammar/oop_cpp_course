\documentclass[10pt,a4paper]{article}
\usepackage[utf8]{inputenc}
\usepackage[T1]{fontenc}
\usepackage{amsmath}
\usepackage{amssymb}
\usepackage{graphicx}
\usepackage[english]{babel}
\usepackage{geometry}
\geometry{top=1cm, left=2cm, top=2cm, bottom=2cm}
\title{Assignment \#6}
\author{Soulimane Mammar}
\begin{document}
	\maketitle
    \section*{Exercise 1}
    Implement a base class \verb|Person|. Derive classes \verb|Student| and \verb|Instructor| from \verb|Person|.A person has a name and a birthday. A student has a major, and an instructor has a salary. Write the class definitions, the constructors, and the member functions \verb|display| for all classes.
    
    \section*{Exercise 2}
    Design an inheritance hierarchy for geometric shapes: rectangles, squares, and
    circles. Provide appropriate constructors for each class. Write the class definitions and implementations of the member functions.
    
    \section*{Exercise 3}
    \begin{enumerate}
    	\item The class \verb|D2| inherits from the class \verb|D1|, which inherits from the class \verb|Base|. To keep \verb|D2| from accessing the public members in \verb|Base|, what access specifier would you use, and where would you use it?
    	\item What is the nature of inheritance with this code snippet? Would your answer be different if Derived were a struct instead?
    	\begin{verbatim}
    	class Derived: Base
    	{
    	    // ... Derived members
    	};	
    	\end{verbatim}
    	\item What is the problem in this code?
    	\begin{verbatim}
    	class Derived: public Base
    	{
    	    // ... Derived members
    	};
    	void SomeFunc (Base value)
    	{
    	    // ...
    	}
    	\end{verbatim}
    \end{enumerate}
    
\end{document}